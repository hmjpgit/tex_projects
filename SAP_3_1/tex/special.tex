\chapter{Anugraha - School for Mentally Retarded}
Anugraha school is located in rural part of Ballari. The story behind its establishment is quite noteworthy. It was built by a mining owner from Ballari. There was a mentally retarded child in his family, who was sent to a nearby rehabilitation center. But the child could not get proper education and counselling there. At that point of time he envisaged the idea of establishing a school which is built with International Standards for Mentally retarded Children. Hence this school came up and is managed by a Christian missionary whose mother center is in Mangalore. 

\section{Background}
Anugraha Trust for the Handicapped was established in 1994. Since the last two decades, we have been working with children having disabilities like visual impairment, hearing impairment, Mental Retardation, Autism, Cerebral Palsy, and Multiple Disabilities. In recent years our Trust’s primary focus of service has been children with Mental Retardation (MR)\footnote{MR - Mentally Retard} and Cerebral Palsy (CP)\footnote{CB - Cerebral Palsy}.\\

Anugraha Trust started its mission for the entire disability sector at two levels, one being Institution Based Rehabilitation (IBR)\footnote{IBR - Institution Based Rehabilitation} and the other being Community Based Rehabilitation (CBR)\footnote{CBR - Community Based Rehabilitation}. IBR focuses on needs of individuals with mental retardation and Cerebral Palsy, whereas CBR focuses on needs of individuals with any kind of disability. When we say individuals, we mean to include children and adults.\\

Within IBR, Anugraha has two institutions; namely a Special School and a Vocational Training Home. CBR includes various completed and ongoing projects of Anugraha, which were or are being carried out in collaboration with other agencies working for the betterment of disabled individuals.

\section{Mission}
Anugraha Trust’s primary motto is the complete rehabilitation of individuals with disability, with a primary focus on individuals with mental retardation. Children with any kind of disability are enrolled in special schools and imparted special education which includes personal grooming skills and functional literacy, with the aim that they can lead ‘normal’ lives like other children.\\

On attaining adulthood, suddenly these individuals are forgotten. Due to their limited IQ\footnote{IQ - Intelligent Quotient}, there is a limitation to the academic teaching that can be assimilated by these individuals and consequently their overall personality development stagnates. After they cross a certain age, they are back to their homes where they become invisible people in the society who are either completely neglected or mistreated and exploited by immoral elements.

\section{Infrastructure}
\begin{enumerate}
    \item Total area of the campus : 9547.84 sq.ft
    \item Building area : 1038.79 sq.ft
    \item Total area of playground : 2932 sq.ft
    \item Total number of classrooms : 22 + 15 = 37. In which 22 are complete for classroom teaching and other 15 are staff rooms, teaching models store room and etc.
    \item Total number of teachers and non - teaching staff : 8 + 6 = 14
    \item Total number of students : 54
\end{enumerate}

\begin{figure}[H]
    \centering
    \includegraphics[width=0.8\linewidth]{Images/special/gate.jpg}
    \caption*{Main Entrance of Anugraha School}
\end{figure}

\section{RCI}
The Rehabilitation Council of India (RCI)\footnote{RCI - Rehabilitation Council of India} has been set up as a Statutory Body under an Act of Parliament and its specific role is to develop, standardize and regulate training programs courses at various levels in the field of Rehabilitation and Special Education.\\ 

It also maintains the Central Rehabilitation Register for qualified Professionals/ Personnel in the area of Rehabilitation and Special Education and promotes Research in Special Education. RCI provides SEMR\footnote{SEMR - Special Education for Mentally Retarded} course for one who are interested to be teachers in special school.

\section{SEMR}
The course in Special Education aims to develop professionals for special education within a broad framework of education in the current millennium. The course will enable pre-service teachers to acquire knowledge, develop competencies and practice skills to impart education to children with special needs. The general objective of the course is to prepare special teachers at pre-primary (Nursery, Kindergarten etc.) and primary (I to IV – lower primary and V to VII – upper primary) levels to serve in the following settings:
\begin{enumerate}
    \item Special schools
    \item Integrated / Inclusive setup
    \item Itinerant programmers
\end{enumerate}

It is Compulsory for every teacher to complete this certification to teach in this school. Teachers who don't have this certificate are trained in APD to make them qualified enough to teach in the special school.

\begin{figure}[H]
    \centering
    \includegraphics[width=0.8\linewidth]{Images/special/RCI.jpg}
    \caption*{RCI Certificate required to work as teachers in special schools}
\end{figure}

\section{APD}
The Association of People with Disability (APD)\footnote{APD - Association of People with Disability} is a non-profit organization based in Bangalore. Founded in 1959, APD works extensively to reach and rehabilitate People with Disability (PwD)\footnote{PWD - People with Disability} from the most underprivileged segments (economically marginalised and deprived communities) of society. \\

Their aim is to create an inclusive society where people with disabilities are accepted into mainstream society - a culture and ecosystem where they can earn, live and sustain with dignity and respect.
\section{Teachers}
There are 8 teaching and 6 Non teaching staff as I have mentioned before. Teachers must have completed SEMR course or at least get trained in APD to work here. This is the list of teachers along with their qualification is given below.


\begin{table}[H]   
\centering
   \begin{tabular}{|c|c|c|}\hline 
       \textbf{Sl.No} & \textbf{Name} & \textbf{Qualification}  \\ \hline 
       1 & Linette Farnandes & BA B.Ed (SEMR)\\ \hline
       2 & Shanta & PUC Diploma in SEMR \\ \hline 
        3 & Anand Yogi & PUC Diploma in SEMR \\ \hline 
       4 & Mary Crasta & PUC Diploma in SEMR \\ \hline 
       5 & Vasanth Kumar & PUC Diploma in SEMR \\ \hline   
       6 & Marsina & ADF Trained \\ \hline 
       7 & Geetha & ADF Trained \\ \hline 
     \end{tabular}
\caption*{List of Teachers and their Qualification}
\end{table}

\section{Students}
There is a special process of selection in the school. Students who want to join this school have to undergo IQ test. It is made to actually confirm that the student is mentally retarded. If the student's IQ is less than 75\% she will be admitted in the school otherwise will be sent to normal schools. \\

\noindent Classes are divided based on the IQ scores of students. They will be classified into groups. Those with less and more IQ in first and last group respectively. A class teacher is appointed to look over the standard proceedings of a class. Here the teacher to students ratio is 1:7\\

At present there are 54 students. As it is a day-scholar school, no classes were conducted during pandemic. As students can't interact with teachers digitally it was a great difficulty for all the faculties. But they somehow managed to visit students' home twice in a month.

\begin{table}[H]
    \centering
    \begin{tabular}{|c|c|c|c|} \hline 
        \multirow{2}{*}{\textbf{Caste}} & \multicolumn{3}{c|}{\textbf{Total}} \\ \cline{2-4}
        & \textbf{Boys} & \textbf{Girls} & \textbf{Total} \\ \hline
        \textbf{Schedule Caste} & 7 & 4 & 11  \\ \hline
        \textbf{Schedule Tribe} & 4 & 2 & 6 \\ \hline 
        \textbf{OBC} & 15 & 4 & 19 \\ \hline 
        \textbf{Minority} & 5 & 2 & 7 \\ \hline 
        \textbf{General} & 6 & 5 & 11 \\ \hline 
        \textbf{Total} & 37 & 17 & 54 \\ \hline 
    \end{tabular}
    \caption*{Number of Students Cateory wise}
\end{table}

\section{Teaching system}
Since there are no classes like the normal schools the teaching system varies. Students with less IQ are taught about colour recognition, shape recognition, face recognition and many more. Once the students are good enough with these skills they are advanced to other classes. Students with good IQ are taught LSRW\footnote{LSRW - Listening, Speaking, Reading and Writing.} way of learning i.e, Listening, Speaking, Reading, Writing. Every 6 months there will be a IQ test in school. If the score is better than their previous one they will be promoted to other levels. \\

Assessment is done in very simple manner. There will be FA1, FA2 and SA1 as in other schools. Same procedure is followed in second half of the academic year too and the weightage is as follows.

\begin{tcolorbox}[coltitle=black,colframe=yellow!90,fonttitle=\sffamily\bfseries\large,title=Division of Marks]
FA 1 for 15 marks\\
FA 2 for 15 marks\\ 
SA 1 for 20 marks
\end{tcolorbox}

\section{Classroom setup - ``U'' Shaped Layout}
We can see that the classroom setup is very unique. There will be 7 students in a classroom. The ``U'' shaped bench is for students. It is made so because of these reasons :

\begin{enumerate}
    \item For smaller classes that want more interaction between the student and educator, a U-Shaped layout is a better option. A U-Shaped desk arrangement encourages discussion and makes it easy for the teacher to observe students and provide one on one help.
    \item Classroom size and number of students can make it difficult to use, for you may not be able to fit a U-Shape pattern in a small room with a large number of students. The layout spreads out children considerably so that it can be hard to address all, and it makes group work harder because the desks can’t easily be moved around.
    \item Advantages: Easy to see and hear everyone in the group. Front of room commands the group’s attention. Unity is created by ganging all the tables together. Openness gives trainees a sense of freedom and encourages participation. Best set up to view audio visual presentations. Works well with role-playing and other physical activities.
    \item Disadvantages: Requires more space than any other configuration. Due to space and learning requirements, the maximum amount of participants should not exceed 24.
    \item Action Zone: Center and at the open end of the ``U''.
    \item Group Involvement: High. It creates a sense of equality within the group.
    \item Tables: Rectangular tables set in a ``U'' configuration. Pie shapes are commonly used at the corners to complete the shape and eliminate the hard edges. Trainer’s table is at the opening of the ``U''.
    \item Accommodates AV: Yes. This configuration is one of the best for visual displays and multimedia presentations. Equipment set at open end of ``U''.
\end{enumerate}

\begin{figure}[H]
    \centering
    \includegraphics[width=0.45\linewidth]{Images/special/classroom.jpg}
    \caption*{``U'' Shaped Classroom setup}
\end{figure}

\section{Art and Craft}
Mentally handicapped children still enjoy art activities and crafts just like any other child. However, these children may need help with certain aspects of the craft due to their impairment. Art activities that are too challenging may only frustrate the child. The goal is to help them enjoy creative expression through art. Many handicapped children can still participate in craft projects that preschoolers enjoy. Every child is an individual and some kids can do more than others.\\

These paper bags were made from students from the waste unused products and card boards. These crafts will be sold commercially and the amount is given to children. In this way school give economical knowledge and develop interest regarding arts and culture in students.

\begin{figure}[H]
    \centering
    \includegraphics[width=0.55\linewidth]{Images/special/special_art.jpg}
    \caption*{Paper bags made by MR students}
\end{figure}

Boys at the Vocational Training Home are involved in various skill development activities. They are trained to make coconut brooms, phenyls and natural acids, floor mats, envelopes and greeting cards, painted frames among other things. The photographs below show the boys engrossed in these activities.

\begin{figure}[H]
    \centering
    \includegraphics[scale=0.21]{Images/special/broom.jpg}
    \caption*{MR students being trained to make Broom`s}
\end{figure}

\section{ADLS}
Activities of Daily Living Skill (ADLS)\footnote{ADLS - Activities of Daily Living Skill}, also referred to as Self Care Skills, play a major role in a child's overall functional growth, confidence and independence. These essential skills include the child's ability to feed themselves using utensils appropriately and to perform toileting, bathing and grooming activities.\\

Problems in this area may be due to an underlying problem, which may include impaired Sensory Integration or diminished Fine Motor or Upper Body Coordination. Children may also exhibit poor motor planning which affects their ability to sequence, time and grade motor activities.\\

We can break these down further into Personal ADLS, which are the "things we normally do every day" such as feeding ourselves, bathing, dressing, grooming, work, homemaking, and leisure and Instrumental activities of daily living (IADLs)\footnote{IADLS - Instrumental activities of Daily living} which are not always necessary for fundamental functioning, but assist an individual to live more independently in a community. \\

The ADLS refer to a series of basic or routine activities performed by individuals on a daily basis in order to take care of themselves, and assist with independent living at home or in the community. There is no fixed list of ADLS, as they depend on the age of the person, their interests, the culture they live in, etc. When we talk about children, we usually refer to the basic activities of daily living which include feeding, eating, dressing, toileting, hygiene (bathing, oral hygiene, etc.) and moving around the house. \\

There are many variations on the definition of ADLs but most organizations agree there are five basic categories.  
\begin{enumerate}
    \item Personal Hygiene such as bathing, grooming and oral care.
    \item Dressing including the ability to make appropriate clothing decisions.
    \item Eating, the ability to feed oneself although not necessarily to prepare food.
    \item Maintaining Continence or the ability to use a restroom.
    \item Transferring oneself from seated to standing and get in and out of bed.
\end{enumerate}

The below image is the best example of the ADLS model. This model is used to make students to get familiar with buttoning and unbuttoning of the shirt and other clothes. In other image we can see all other ADLS materials which are used to teach.\\

\includegraphics[width=0.4\linewidth]{Images/special/TLM3.jpg} \hfill
\includegraphics[width=0.45\linewidth]{Images/special/TLM.jpg}

\section{Activities}
Anugraha Trust for the Handicapped undertakes various activities at the special school and the adult vocational home so that the students at both these institutions learn and assimilate new concepts, develop their skills as well as attain a certain level of maturity. At the school special education is imparted in four classes namely pre-primary, primary, trainable and pre-vocational.  The programme offered comprises therapeutic management, special education and pre-vocational training. At the adult vocational home training is provided in various vocations according to individual capacities. Vocations include coconut broom making, flour grinding, jewellery and gift item making.\\

\begin{figure}[H]
    \centering
    \includegraphics[width=0.51\linewidth]{Images/special/voC.png}
    \caption*{Students performing Lezim`s}
\end{figure}

\section{Grants by the Government}
SSA\footnote{SSA - Sarva Shiksha Abhiyan} provides grants to both types of schools- which are run by the Department of Education and which supported/aided by the Department of Education. This is completely different from Aided Schools. In Aided schools government takes control over most of the administration of school. In Granted schools only Financial support is provided by government. \\

They are provided with School grant, Teacher grant for preparation of teaching learning material. In addition to that the schools of the Department having own building are provided with school grants for repair and maintenance and grant for new Civil works. This School got grant in the year 2019 after 8 years of successful functioning, the types of grants this schools receives are,
\begin{enumerate}
    \item Teaching Learning Equipment (TLE)\footnote{TLE - Teaching Learning Equipment} Grant 
    \item Schools grant :
    \begin{enumerate}
        \item For the maintenance of School records and contingency expenditure.
        \item For the preparation and implementation of School Development Plan (a plan chalked out for the development of the school based on its needs).
        \item For the preparation of teaching learning materials.
    \end{enumerate}
    \item Maintenance grant :
    \begin{enumerate}
        \item For protection of school building and maintenance of the school like whitewashing, minor repair works.
        \item Payment of electricity and telephone bills, provision of drinking water facility,  maintenance of  toilets and cleanliness of school and its campus.
        \item Installation of incinerators and purchase of sanitary napkins for girl students under \textbf{SUCHI} Program studying in  Schools.
        \item If there is savings after implementing the above activities, the amount may be used for Teacher Grant.
    \end{enumerate}
    \item Teacher grant : Every teacher of the School is given Rs. 500/- per year as teacher grant for the preparation of teaching-learning materials.
\end{enumerate}

\begin{figure}[H]
    \centering
    \includegraphics[width=0.5\linewidth]{Images/special/special_govt_franted.jpg}
    \caption*{Anugraha School Granted by Govt. of Karnataka}
\end{figure}

\section{Sports and Cultural Events}
Although great strides and progress have been made in physical education, recreation and sports programs for special population in the last 10 to 15 years, much remains to be done. Especially for the students with disabilities. There is no such importance for that, but in this school it have played much important role. Student even participated in national level events. The below image is the Football ground made for mentally retarded students based on the international standards. Along with this there is basketball court and common activities ground too. 

\begin{figure}[H]
    \centering
    \includegraphics[width=0.6\linewidth]{Images/special/playground.jpg}
    \caption*{Playground in Anugraha School}
\end{figure}

Once in a year there will be a School Day event. All other national festivals are celebrated very happily here. In spite of their difficulties students have made some wonderful acts. 

\begin{figure}[H]
    \centering
    \includegraphics[width=0.7\linewidth]{Images/special/act.jpg}
    \caption*{Drama enacted by Special students}
\end{figure}

Certainly we can see more emotions in them than normal students. Instead of pointing out them as \textit{Disabled students} we can name them as \textbf{Special Differently Abled students} who are really special in all terms than the students studying in normal schools.
\cleardoublepage