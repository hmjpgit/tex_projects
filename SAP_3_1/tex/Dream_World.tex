\setcounter{page}{1}
\pagenumbering{arabic}
\chapter{Dream world School}
\section{The Background}
Human life is a blend of social and spiritual objectives as much as it is a blend of political, economic and empirical-scientific-rational objectives. That is why all schools have a challenging role in shaping their students’ intellectual and ethical capabilities and values.\\

However,  they believe that that blend must also include emotional as well as spiritual aspects of human existence. Every student comes to them to be educated in how to acquire factual and technical information and knowledge as well as a responsible spiritual consciousness. What begins to be called research in specialized fields of human activity in later years requires development of an inquiring mind. With that understanding and conviction, Dream World School was established in 1998 under the aegis of the VAG India Trust* with the motto “Educate to Research Life”.\\

Life is nurtured both before and after birth. Continuing that analogy, schools are expected to serve as “nurseries” for young minds and bodies for their steady and positive development into responsible as well as adventurous citizenship of an ever-evolving country and world.\\

These concepts constitute the foundation of the functioning of Dream World School. That is why the School has six important “organs” that provide nourishment for both minds and bodies: PRAGYĀNA, PRASTHĀN, SAGA-7, [DP]$_2$, FSSA and PASCUL.* They mean to nurture every student.

 \section{Historical Sketch}
In the first decade of its existence, Dream World School worked with the motto “Industry”, the way an enterprise works, with the objective of producing and developing core qualities of young personalities such as time-management, responsibility and accountability, (re)searching skills and communication skills.\\

In the second decade, Dream World School aspired to grow towards the motto “Temple”. For at this stage  they began to pay special attention to spirituality to balance and brighten students’ thought processes. They introduced training in activities like meditation, celebration of festivals, working on the principle of the role of ‘Gurus’ as repositories of information, knowledge and wisdom and instilling moral values that might promote aspiration to participate in a divine environment.\\

After having introduced, during their first two decades, the above aspects into their programme to help young people acquire knowledge and experience, next  they introduced the third stage of growth the process of Learning to Identify Problems and Work toward their Solutions. That applies to the School as it does to students. Accordingly, the school prepares case studies of as many individual students as possible to try to discover learning problems so as find solutions to them. The “problems” may be in the academic field or in the domains of discipline, self-control or moral and spiritual questions that govern behaviour. Once discovered, these solutions are applied methodically according to the needs and behaviour patterns of individual students.\\

 That is how, in the third decade of their existence,  they are determined to foster the objectives of the first two decades and focus on working toward “Solutions” to facilitate students in their education.  they hope that, as a result, when they graduate they are able to take their places in the life of their quickly developing nation and the evolving world, fulfilling their objectives in all domains of human life.\\
 
  \section{Mission}
 Given the above objectives and aspirations, the school believes its Mission is to prepare young people to meet the needs and demands of human society locally, nationally and on the world stage with a constructive view of the future. The following objectives follow from this mission:
\begin{enumerate}
\item All-round development of students to enable them to become assets for the Nation.
\item Individual attention to students to identify, consolidate and sharpen their learning capabilities.
\item Diagnosis of individual students’ problems by specific case studies and creation of rational solutions to them.
\item Developing skills in inquiry, research and contextualisation.
\item Providing experiential learning.
\item Inculcating values not imposed but acquired on the basis of analysis by inquiring minds.
\item Bringing spiritual consciousness through programmes such as Pragyana.
\item Developing the full range and variety of skills required in continuing education.
\item Striving for excellence in all aspects of education.
\item Inculcating physical and mental discipline.
\end{enumerate}
\section{Vision}
\begin{center}
\textbf{Educates to Research ‘Life’}
\end{center}

To bring about self-awareness through education so as to develop inquiring minds to fulfil man’s ultimate goal of knowing himself even as he serves human society and the natural world efficiently while upholding cultural and moral values.
\section{Basic Details}
\begin{table}[H]
    \centering
    \begin{tabular}{|l|l|} \hline
    \textbf{School Name} & Dream World School, Ballari \\  \hline
    \textbf{Principal} & Ms. Parimala CM\\ \hline
    \textbf{Affiliated to} & Central Board of Secondary Education (CBSE)\\ \hline
    \textbf{Affiliation Code} & 830108\\ \hline
    \textbf{Address} & Dream World School, Kappagal Road, Ballari\\ \hline
    \textbf{Landmark} & Canal Distribution No. 14\\ \hline
    \end{tabular}
    \caption*{Basic Details of school`s Administration}
\end{table}
\normalsize

\section{Infrastructure Details}
\begin{longtable}{|l|c|} \hline 
Total number of building blocks & 2\\ \hline 
Total area of school in square metres & 12146\\ \hline 
Total number of playgrounds & 1\\ \hline
Total area of playground in square metres & 6601\\ \hline 
Total number of rooms & 48\\ \hline 
Total number of small-sized rooms & 15\\ \hline 
Total number of medium-sized rooms & 25\\ \hline 
Total number of large-sized rooms & 8\\ \hline 
Total number of Male rest room & 1\\ \hline 
Total number of Female rest room & 2\\ \hline 
Number of Girls' toilet & 18\\ \hline 
Number of Boys' toilet & 12\\ \hline 
Number of toilets for differently abled persons & 1\\ \hline 
Number of washrooms for Female staff & 8\\ \hline 
Number of washrooms for Male staff & 4\\ \hline 
Total number of Libraries & 1\\ \hline 
Number of Laboratories & 5\\ \hline 
Number of water purifiers/ROs & 1\\ \hline 
Number of digital classrooms & 5\\ \hline 
Does the school has hostel facility & NO\\ \hline 
Does the school has CCTV cameras installed? & YES\\ \hline 
Is the school examination center of CBSE? & NO\\ \hline 
Total number of computers in all computer lab & 42\\ \hline 
Does the school has a boundary wall? & YES\\ \hline 
Does the school has clinic facility? & NO\\ \hline 
Does the school has a strong room? & YES\\ \hline 
Does the school has sports facility? & YES\\ \hline 
\caption*{Thorough details of School`s Infrastructure and Classrooms}
\end{longtable}


\section{Strength of students Class wise}
\begin{figure}[H]
    \centering
    \includegraphics[scale=0.6]{Images/Dream_World/Strength.png}
\end{figure}

 
 \section{Mode of Assessment}
 \begin{table}[H]
\footnotesize
\begin{tabular}{|c|c|c|c|c|c|c|}
\hline
\multicolumn{5}{|c|}{\textbf{Internal Assessment}}                                                                                                                                                                                                                                                                              & \multicolumn{2}{c|}{\textbf{Term End Examination}}                                                                                                                                                                                 \\ \hline
\multicolumn{3}{|c|}{}                                                                                                                                                                                                                                 & \multicolumn{4}{c|}{\textbf{Weightage of Marks}}                                                                                                                                                                                                                                                   \\ \hline
\multicolumn{1}{|c|}{\textbf{Sl. No}}             & \multicolumn{1}{c|}{\textbf{Area}}                                                                            & \multicolumn{1}{c|}{\textbf{Task}}                                                                            & \multicolumn{1}{c|}{\textbf{Term One}} & \multicolumn{1}{c|}{\textbf{Term Two}} & \multicolumn{1}{c|}{\multirow{9}{*}{\begin{tabular}[c]{@{}c@{}}Term one \\ Examination\\ (35+5) Marks\end{tabular}}} & \multirow{9}{*}{\begin{tabular}[c]{@{}c@{}}Term two \\ Examination \\ (35 + 5) Marks\end{tabular}} \\ \cline{1-5}
\multicolumn{1}{|c|}{\multirow{2}{*}{1}} & \multicolumn{1}{c|}{\multirow{2}{*}{\begin{tabular}[c]{@{}c@{}}Multiple\\ Assignment\end{tabular}}}  & \multicolumn{1}{c|}{Quiz}                                                                            & \multicolumn{1}{c|}{1}        & \multicolumn{1}{c|}{1}        & \multicolumn{1}{c|}{}                                                                                                &                                                                                                    \\ \cline{3-5}
\multicolumn{1}{|c|}{}                   & \multicolumn{1}{c|}{}                                                                                & \multicolumn{1}{c|}{Concept (Mind) Map}                                                              & \multicolumn{1}{c|}{1}        & \multicolumn{1}{c|}{1}        & \multicolumn{1}{c|}{}                                                                                                &                                                                                                    \\ \cline{1-5}
\multicolumn{1}{|c|}{\multirow{3}{*}{2}} & \multicolumn{1}{c|}{\multirow{3}{*}{Portfolio}}                                                      & \multicolumn{1}{c|}{Art Integrated Project}                                                          & \multicolumn{1}{c|}{1}        & \multicolumn{1}{c|}{-}        & \multicolumn{1}{c|}{}                                                                                                &                                                                                                    \\ \cline{3-5}
\multicolumn{1}{|c|}{}                   & \multicolumn{1}{c|}{}                                                                                & \multicolumn{1}{c|}{Asignment}                                                                       & \multicolumn{1}{c|}{-}        & \multicolumn{1}{c|}{1}        & \multicolumn{1}{c|}{}                                                                                                &                                                                                                    \\ \cline{3-5}
\multicolumn{1}{|c|}{}                   & \multicolumn{1}{c|}{}                                                                                & \multicolumn{1}{c|}{Notebook Submission}                                                             & \multicolumn{1}{c|}{1}        & \multicolumn{1}{c|}{1}        & \multicolumn{1}{c|}{}                                                                                                &                                                                                                    \\ \cline{1-5}
\multicolumn{1}{|c|}{3}                  & \multicolumn{1}{c|}{\begin{tabular}[c]{@{}c@{}}Subject \\ Enrichment\\  Activity (SEA)\end{tabular}} & \multicolumn{1}{c|}{Practical Lab Work}                                                              & \multicolumn{1}{c|}{3}        & \multicolumn{1}{c|}{3}        & \multicolumn{1}{c|}{}                                                                                                &                                                                                                    \\ \cline{1-5}
\multicolumn{1}{|c|}{\multirow{2}{*}{4}} & \multicolumn{1}{c|}{\multirow{2}{*}{Periodic Tests}}                                                 & \multicolumn{1}{c|}{\begin{tabular}[c]{@{}c@{}}PT 1 and PT 2\\ (Best of the two tests)\end{tabular}} & \multicolumn{1}{c|}{3}        & \multicolumn{1}{c|}{-}        & \multicolumn{1}{c|}{}                                                                                                &                                                                                                    \\ \cline{3-5}
\multicolumn{1}{|c|}{}                   & \multicolumn{1}{c|}{}                                                                                & \multicolumn{1}{c|}{PT 3}                                                                            & \multicolumn{1}{c|}{-}        & \multicolumn{1}{c|}{3}        & \multicolumn{1}{c|}{}                                                                                                &                                                                                                    \\ \hline
\multicolumn{1}{|c|}{}                   & \multicolumn{2}{c|}{\textbf{Total Weightage of Marks}}                                                                                                                                                               & \multicolumn{1}{c|}{10}       & \multicolumn{1}{c|}{10}       & \multicolumn{1}{c|}{40}                                                                                              & 40                                                                                                 \\ \hline
\multicolumn{1}{|c|}{}                   & \multicolumn{2}{c|}{\textbf{Term One + Term Two}}                                                                                                                                                                    & \multicolumn{4}{c|}{(10 + 40) + (10 + 40) = 100}                                                                                                                                                                                                                                          \\ \hline
\end{tabular}
\end{table}


\normalsize
\section{Prasthan Exhibition}
‘Prasthan Exhibition’ is a Science Exhibition to showcase the talents of students and to promote creativity. In order to endorse research skills and ignite innovative ideas among students, they organize ‘Prasthan Exhibition’ every year in their school. This is a great opportunity for their students to drive their competencies in thinking, creative and scientific skills. They believe that one should learn science by practical or experimental approach rather than mere reading textbooks or theory.\\

This exhibition cultivates scientific approach and reasoning skills among the aspirants. This is also a platform for the students to exhibit their models in CBSE Science Exhibition in Regional Level at Bangalore, later in National Level at New Delhi. Bagging two first prizes at National level CBSE Science Exhibition 2017–2018 in two different themes is a significant milestone for their students’ achievement. This achievement is an inspiration and a benchmark for all other students to dream their goals and strive to fulfil them.
\section{Co-Curricular Activities}
\begin{figure}[H]
    \centering
    \includegraphics[scale=0.23]{Images/Dream_World/122-converted-1.jpg}
\end{figure}
\section{Biodiversity}
Children go to school to learn. But, what do they learn? Schools are recognized as not only places of formal education, but also as critical settings to foster health, development and overall well-being of pupils and their families. School gardens can be defined as cultivated areas around or near schools, maintained (at least partly) and used by pupils and teachers in different manners.\\

Comprehensive models of action are currently being promoted, whereby schools can develop multiple-win situations and positive outcomes in learning performance, food security, nutrition, rural development, local economy and lifestyle practices and habits.\\

The Objective of SAVE TREES Project is, Children can : 
\begin{enumerate}
\item Learn how to grow a variety of fresh and nutritious foods, and how to improve their diets with home-grown foods; 
\item see how these foods link with a healthy (and economical) diet;
\item taste new foods and learn basic skills in preparing fresh foods from the garden;
\item change their perceptions of the environment, ecosystems, nutrition and food waste; 
\item develop team work skills and a sense of 
responsibility;
\item learn to value the work of their parents 
and other people on planting the fruits 
and vegetables.
\end{enumerate}

There are 60+ trees planted in school during its Initial Days and now they are grown tall. I have listed 2 of them, for entire list of the trees, \href{http://www.dreamworldschool.com/PDF/Save-Trees.pdf}{Click here SAVE TREES - Dream World School} 

\begin{tcolorbox}[coltitle=black,colframe=yellow!90,fonttitle=\sffamily\bfseries\large,title=Sapodilla]
Name of Plant Species: Manilkara zapota\\
Height in cm: 30.99 (as on 14.09.2018)\\
Width in cm: 19.05 (as on 14.09.2018)\\
Date of Plantation: 20.08.2015\\
Name given to it: Brigadier Mohammad Usman\\
\end{tcolorbox}

\begin{tcolorbox}[coltitle=black,colframe=yellow!90,fonttitle=\sffamily\bfseries\large,title=Tamarind]
Name of Plant Species: Tamarindus indica \\
Height in cm: 15.24 (as on 14.09.2018) \\
Width in cm: 35.56 (as on 14.09.2018) \\
Date of Plantation: 20.08.2015 \\
Name given to it: Major Sundeep Unnikrishnan \\
\end{tcolorbox}

There are some animals being raised in the School`s Campus and all kind of care is taken about them. There is resting shed and Required amount of fodder and supplement food is provided. Veterinary doctor visits regularly and frequent Health Checkup is done.

\begin{figure}[H]
    \centering
    \includegraphics[scale=1.25]{Images/Dream_World/Animal2.png}
\end{figure}

\section{Alumni}
\begin{figure}[H]
    \centering
    \includegraphics[scale=0.75]{Images/Dream_World/Ashwija.png}
    \caption*{The Felicitation of B.V. Ashwija I.A.S. - The Pride of Dream World School}
\end{figure}

It’s a privilege for a school and its teachers, when their students stand out to face the society with the colours of success. This feeling of pride and honour was experienced by Dream World School whenone of its students who reached her line of success – Ms. Ashwija B. V. was invited for the programme of Felicitation by the School. \\

Ashwija B. V., was the student of Dream World School and she was one among the first batch of outgoing students of their School in 2007-2008. Drawing from some of the most pivotal points of her school life, she always showed great potential at school. She was not only an excellent student but also a leader who could make her team win. Every aspect of her school life was driven by never ending competitive spirit and determination. \\

Ashwija has qualified her I.A.S. exams in 2018 clearing all the three phases at a stretch, with 423rd rank, which she had dreamt of and has been working for since her schooling.Probably she has continued to be determined in her life which has brought her to this height of success. Isn’t it something great! \\

Dream World felt the pride in sharing the proud moment of felicitating their Student Ms. Ashwija I.A.S., for her achievements. Her achievement might ignite the power of determination in the minds of the young learners to reach their goal. These glimpses of success is the saga of Dream World. \\

