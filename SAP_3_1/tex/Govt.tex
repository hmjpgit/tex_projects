\chapter{Government High School Indiranagar}
\section{Vision}
Let us all Learn - Let us all Grow :
''Provision of relevant and useful Elementary Education of satisfactory quality for all children with full concern for social and gender equity and regional parity and with vibrant participation of the community in the management of schools''.

\section{Mission}
To provide access and to enroll and retain all the children between 6 to 14 years of age in schools including specially abled and to impart quality education inbued with universal human values equipping them to contribute constructively to society through empowering teachers and enlighten communities.

\section{History}
This School was established in the year 2006 and is set mark its 15 years of successful running on August 15 this year. The is made for the students who are from the Economically Weaker Background. The School is located in the Industrial Area of Indiranagar,  Ballari city. The full name of the school is \textbf{Sarkari Prouda Shaale}.\\

\begin{figure}[H]
    \centering
    \includegraphics[width=0.69\linewidth]{Images/Govt/Main.jpg}
\end{figure}

The School presently teaches to class 9 and 10 students. The school is owned by the Government of Karnataka hence the Karnataka State Board (KSEEB)\footnote{KSEEB - Karnataka State Board} curriculum is being followed. Anyone from any background can join this School at any time.\\

\section{Infrastructure}
The school is located in 850 square metres area with 6 rooms. There is 1 staff room and a Head Mistresss room. The playground is in 0.5 acre of land which is quite less compared to the number of students in the School. There need to be much emphasis on building since it has been constructed 15 years ago, and no renovation has been done for this building till now. \\ 

Class rooms are completely damaged. There is no hope for the school to depend on those classrooms. Government has not taken any steps to renovate or to build a new one. 

\section{Drinking water and Toilet facility}
More than 180 students staying there on the campus and sadly there is no Drinking Water Facility in school till date. School Head Mistress have been approaching higher authorities from past 6 years and government reluctantly neglect this matter. Since there is no go even school have stopped approaching higher authorities.\\

\begin{figure}[H]
    \centering
    \includegraphics[scale=0.025]{Images/Govt/Jamipol.jpg}
\end{figure}

Since it is a government school the infrastructure for sanitation was very poor. In 2021 an organisation called \textbf{Jamipol} which is powered by \textbf{Akshaya Pathra} foundation have come forward under their Corporate Social Responsibility (CSR)\footnote{CSR - Corporate Social Responsibility} they donated 2 Water bottles, 2 Masks and 1 Hand Sanitizer to every student and Faculties.   

\section{Students}
\begin{table}[H]
    \centering
    \begin{tabular}{|c|c|c|c|c|c|c|c|c|c|c|c|} \hline
        \textbf{Category} & \multicolumn{2}{c|}{\textbf{General}} & \multicolumn{2}{c|}{\textbf{OBC}} & \multicolumn{2}{c|}{\textbf{SC}} & \multicolumn{2}{c|}{\textbf{ST}} & \multicolumn{2}{c|}{\textbf{Total}} & \textbf{Grand Total} \\ \hline
         \textbf{Class} & B & G & B & G & B & G & B & G & B & G & - \\ \hline
         \textbf{9th} & 1 & 0 & 27 & 13 & 13 & 8 & 17 & 15 & 58 & 36 & 94\\ \hline
         \textbf{10th} & 0 & 0 & 25 & 15 & 14 & 19 & 18 & 11 & 57 & 35 & 92 \\ \hline 
    \end{tabular}
    \caption*{Number of students in every class divided based on the category}
\end{table}

There are 186 students in total. We can see that number of girl students is less comparing to boys, when I enquired for more information Head Mistress said that its same every Academic year. We can regulate this by Advertising about Girl child education in Televisions.

\begin{figure}[H]
    \centering
    \includegraphics[scale=0.6]{Images/Govt/Strength.png}
\end{figure}

\section{Teachers}
There are 7 Teaching staff out of which 1 is male and the rest are female teachers. 4 non-teaching staff are appointed as Bisi Oota (Mid-day Meal)\footnote{MDM - Mid Day Meals} managing members. There is a need of attenders and cleaners in the school. Here is a list of teachers along with their qualification-

\begin{table}[H]
    \centering
    \begin{tabular}{|c|l|c|c|} \hline
        \textbf{Sl. No} & \textbf{Teachers Name} & \textbf{Qualification} & \textbf{Subjects Handled} \\ \hline 
        1 & Mrs. Pinjar Honoor Bi & KES & Head Mistress \\ \hline 
        2 & Mrs. R Uchhangamma & M.Sc, B.Ed & Physics, Chemistry, Maths\\ \hline 
        3 & Mrs. P Shreelatha & MA, B.Ed & Hindi\\ \hline
        4 & Mrs. Vijayataara & BA, B P.Ed & Physical Education\\ \hline
        5 & Mr. Naageshwara Rao & Diploma in Fine Arts & Art and Craft\\ \hline
        6 & Mrs. Mary Roshini & MA, B.Ed & Kannada \\ \hline
        7 & Mrs. Sunitha MC & MA, B.Ed & English \\ \hline
    \end{tabular}
\end{table}

\section{Mid-Day Meals (MDM)}
Mid-Day Meal program is the Government of India's flagship program to achieve the Universalisation of Elementary Education (UEE)\footnote{UUE - Universalisation of Elementary Education} and is being implemented in partnership with State Government to cover the entire Karnataka. \\
Food is provided every afternoon for both students and teachers. The images given below show the quantity of the ingredients to be added to the food and the schedule of the food items provided.
\begin{figure}[H]
    \centering
    \includegraphics[scale=0.0525]{Images/Govt/MDM.jpg}
\end{figure}

\section{Teaching and Scheme of Evaluation}
\begin{enumerate}
    \item From Class 9 and 10 \textbf{CCE}\footnote{CCE - Continuous and Comprehensive Evaluation} (Continuous and Comprehensive Evaluation) is followed. This type of teaching system is the traditional one and there will be some important aspects like lesson plan, teaching hours, scheme of evaluation and etc. Every subject teacher will submit his/her lesson plan before each and every class and get it corrected by the Head Mistress.\\
    
    Lessons are taught interactively. Traditional Blackboard teaching is used in this school. The teacher arranges a unit test after completion of each and every lesson. This is considered as one of the best effective methods in CCE learning to evaluate the student. In final report card it is given some importance.

\item Apart from this there will be 2 FAs (Formative Assessment) and 1 SA (Summative Assessment) in a semester which are much important than class tests. 

\begin{tcolorbox}[coltitle=black,colframe=yellow!90,fonttitle=\sffamily\bfseries\large,title=Division of Marks]
 FA 1 of 15 marks\\
 FA 2 of 15 marks \\
 SA 1 of 20 marks
\end{tcolorbox}
\end{enumerate}
    
\section{Exam and Evaluation}
\begin{enumerate}
\item Exam is \textbf{Conducted Based on Continuous and Comprehensive 
Evaluation} (CCE) system. Blueprint is made by subject teachers and 
later approved by the HeadMistress for Summative Assessments at the end of 
the semester (SA 1 and SA 2).
\item Invigilators will be teachers of subjects other than the subject of the test. Seating arrangements are made in such a way that in a bench 3 
students are allowed to sit who are from different classes. 
\item Subject teachers set the question paper and get it approved by HeadMistress before the exam. 
\end{enumerate}

\section{Report Card} 
All government schools in Karnataka have the same Report card layout. 1 st Page of the Report card includes details of the students such as name, class, date of birth, admission number, parents name, address, and students' photo. It also includes height, weight, and blood group of students. \\
School Details such as the Academic year and School Registration code will also be included. In next Page, it contains the academic achievements of the student which has subject wise break-up and their respective grades. 

\begin{figure}[H]
    \centering
    \includegraphics[width=0.9\linewidth]{Images/Govt/report_card_govt.jpg}
    \caption*{Report Card}
\end{figure}

Achievements in Co-Curricular activities are also included as per the Grading system mentioned in this Report. At the end of the Report, there will be mentioning of student's promotion to next class. It will be duly signed and sealed by School HeadMistress and Will be distributed to students.

\section{Revision and Remedial Classes}
Wherever a student is lagging behind in academics, the teacher recognizes them and makes a list. Later these students are counselled by the respective subject and class teachers. If they find that there are Academic Related Issues, they will go for revision classes. All these procedures will happen during the Bridge Courses. If a student is found to be not that responsive then there will be Special Remedial Classes after school hours to improve his/her Academic Performance. 

\section{School Development Monitoring Committee (SDMC)}
\begin{enumerate}
\item To ensure community ownership and community participation in education, the Government have evolved this system of having a School Development and Monitoring 
Committee for each and every Government school. 
\item Apart from others, the main members of this committee will be 9 parents whose children are studying in the given school. 
\item The SDMC has been given necessary powers and functions for ensuring that the schools are managed better. Most of the issues relating to the academic aspects and developmental activities of the schools are addressed by SDMCs. 
\item The Head Mistress of the school functions as the Secretary of the SDMC.
\item The present system has been evolved by the issue of executive orders of the Government of Karnataka, and action is underway to continue the system of SDMC and even make it more effective. A proposal to make these SDMCs as part of the Panchayati Raj Institutions - by making them sub-committees of the respective Gram Panchayaths - is under consideration by the Government, in coordination with the Rural Development and Panchayath Raj Department.
\end{enumerate}

\begin{figure}[H]
    \centering
    \includegraphics[width=0.8\linewidth]{Images/Govt/SDMC.jpg}
    \caption*{Teachers Interacting with SDMC Officials}
\end{figure}

\section{Role of Community Participation in Schools}
\textbf{Parents and Teacher’s Interaction} -  After each Assessment is completed, there will be a Parent-teachers interaction to inform parents about their wards' academic progress. Most parents won’t attend this. So teachers will visit those students' homes whose academic performance is very low and inform parents the same.
\section{Events and Sports}
Sports are given a lot of importance in this school. As there is one more government school nearby this school children use their playground for all the sports activities. There have not been any such remarkable sports achievements but there is a lot of emphasis on Physical Education in this school. 

Science events are also an important part of this school. Some of the events are 
\begin{enumerate}
    \item \textbf{Pratibha Karanji} \\
    This event is organized once a year by the Government of Karnataka to encourage Co-Curriculum Activities for the students. There will be many competitions like Clay Modelling, Fancy Dress, Dance, Singing and many more. Students from this school have achieved district-level prizes.
    \item \textbf{TLM Mela} - Teaching Leaning Materials Mela \\
    This event is conducted by teachers to give insight into the teaching materials they use and how they are used. Especially science and math-related models are showcased here. Students also prepare some of the Models and there will be competition for that. Teachers train them and INSPIRE Award is given to winners from the government.  
\end{enumerate}

\section{Government schemes}
\begin{enumerate}
    \item Vitamin tablets like Pholic acid, albendazole are provided to students to fight malnutrition among them.
    \item Once a month, there will be a health check-up by a government ENT specialist. The check-up includes ENT, Tooth cavity, height, and weight. 
    \item  Minority students get Scholarships from the Government. 
\end{enumerate}
\section{Dropout Students}
Dropping out means leaving high school, college, university, or another group for practical reasons, necessities, or disillusionment with the system from which the individual in question leaves.\\

The School Head Mistress and BEO\footnote{BEO - Block Education Officer} of that sector will take the initiative in bringing back those children to school. Those students are recovered from home after a long absence. They are trying to attend class right now. Department is striving very hard to get those students to normal streamline. 

\begin{figure}[H]
    \centering
    \includegraphics[scale=0.35]{Images/Govt/Droupouts.jpg}
    \caption*{Education Department Officers visiting homes of Dropout Student`s}
\end{figure}

\section{Schools in the times of pandemic}
\begin{enumerate}
    \item This novel pandemic disrupted the normal functioning of schools and their activities. Due to lockdowns schools were shut. So new methodologies were adopted to facilitate the teaching and assessment of students. Online classes and assessments replaced conventional learning.
    \item Private schools did all the academic activities in virtual mode - conducting classes, submission of assignments in typed/scanned format, etc. Those students who couldn't afford smartphones faced problems by this.
    \item Meanwhile, the government started \textbf{Vidyagama} and \textbf{Vathara Shaale} initiatives along with online classes which helped poor children who can't afford electronic gadgets. 
    \item Teachers went to students' houses, students and teachers gathered in open public places like temples, groupings near home, and academic activities were done. 
    \item Teachers used to assign some works to students there and marks were awarded for the same whereas, for higher classes, activities and assignments were given based on the syllabus for formative and summative assessments which students submitted in the next class. Marks were awarded for the same.
    \item  These internal marks along with student's previous year's academic performance was used to give results. 
\end{enumerate}

\begin{tcolorbox}[coltitle=black,colframe=yellow!90,fonttitle=\sffamily\bfseries\large,title=Online Classes]
Some schools conducted online classes on various platforms like zoom, Google meets, Microsoft teams, DD Chandana classes were also telecasted simultaneously.
\end{tcolorbox}

\begin{figure}[H]
    \centering
    \includegraphics[scale=0.25]{Images/Govt/Online.jpg}
\end{figure}

In some schools, formative assessments were based on activities and summative assessments were based on assignments. On the other hand, some schools used internal marks for activities and precious year`s performance to give the final result of students. 

\section{ICT Integration}
\begin{figure}[H]
    \centering
    \includegraphics[scale=0.35]{Images/Govt/ICT.jpg}
    \caption*{Students Watching Online Workshop on Yoga}
\end{figure}

\begin{enumerate}
\item Objectives of ICT in Schools
\begin{enumerate}
    \item  Establish an enabling environment to promote the usage of ICT in
schools.
\item Enhance the learning levels of students in Mathematics, Pure Sciences,
Social Sciences, Language and numerous Extra- Curricular activities.
\item Promote critical thinking and analytical skills by developing self learning
\item Enable students to acquire skills needed for the digital world for higher studies and gainful employment.
\item Build capacity in teachers to upgrade their learning and teaching skills by
using ICT tools.
\end{enumerate}
\item Impact of ICT programme
\begin{enumerate}
\item  Improvement of enrolment and attendance
\item Increase of computer literacy among students and teachers
\item Larger number of computer-trained teachers
\item Enhanced computer-aided learning by students
\item Significant increase in pass percentage in Tenth Standard Public Exam
\end{enumerate}
\end{enumerate}