\renewcommand{\chaptername}{}
\chapter{Inclusiveness Observed in Schools}
An estimated 240 million children worldwide live with disabilities. Like all children, children with disabilities have ambitions and dreams for their futures. Like all children, they need quality education to develop their skills and realize their full potential.\\

Yet, children with disabilities are often overlooked in policymaking, limiting their access to education and their ability to participate in social, economic and political life. Worldwide, these children are among the most likely to be out of school. They face persistent barriers to education stemming from discrimination, stigma and the routine failure of decision makers to incorporate disability in school services. Robbed of their right to learn, children with disabilities are often denied the chance to take part in their communities, the workforce and the decisions that most affect them.

Disability is one of the most serious barriers to education across the globe. Inclusive education allows students of all backgrounds to learn and grow side by side, to the benefit of all.

    
\begin{center}
\textbf{Every child has the right to quality education and learning}
\end{center}

\section{Dream World School}
As of my observation this was school was not Inclusive. Although there has been lot of Infrastructure setup for Physically disabled students like Washroom for Disabled students and etc, Yet there has not been any enrollment by Physically Disabled students in this School. And i also think that the school management have decided not to make any such enrollments. When I further enquired it there was very meagre response from the School authorities.\\

Although students from all caste have been enrolled in school there was also some rules which should be followed by students like they have chant mantras of Lord Tirupati Venkatheshwara swami and many more.\\

By above observation I conclude that the school has taken very less initiative with respect to Inclusiveness of the School.

\section{Government High School Indira Nagar}

Since it is a government School it is definitely Inclusive School. Students from all caste, religion and economical backgrounds are admitted here without any partiality. Students with Disability are enrolled each year and additional facilities required for students are being provided from school. Some of them are Separate Washroom for Dialed students, special arrangement in Classroom setup since they cant sit for long hours and many more.\\

There are special arrangements made by Block Resource Centers (BRC)\footnote{Block Resource Centers} officers where they visit home of those students who cant even make to school and take care of their education. All the Scholarships are given to them which is not only for education but also for their livelihood.\\

They are brought to school for some events and given special attention there too. By this I concluded that this school is very much Inclusive.

\section{Anugraha School}
This School is specially for Mentally retarded students, so there is nothing to assess much here. This is School for Specially Mentally Retarded so no Inclusiveness was Observed.