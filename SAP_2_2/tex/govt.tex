\setcounter{page}{1}
\section{Government Higher Primary Ambedkar School}
This School was established in the year 1981 and is set mark its 40 years of successful running on August 15 this year. The is made for the students who are from the Economically Weaker Background. The School is located in the Industrial Area of Indiranagar,  Ballari city. The full name of the school is \textbf{Sarkari Hiriya Prathamika Ambedkar Shaale}.\\
\noindent The School presently teaches from classes 1 to 8. The school is owned by the Government of Karnataka hence the Karnataka State Board (KSEEB) curriculum is being followed by it. Anyone from any background can join this School at any time. It consists of 301 students in total comprising 152 boys and 149 girls. There are 10 teaching and 4 non-teaching staff. 
\subsection{Mission}
\begin{itemize}
    \item To give quality education to the underprivileged children. 
    \item To act as a Cluster Level Government School in Ballari city.
\end{itemize}
\subsection{Objectives of School}
\begin{itemize}
\item Impart quality education by employing latest techniques
\item Emphasise harmonious human relationships and high moral standards through active participation in curriculum development 
\end{itemize}

\begin{figure}[H]
    \centering
    \includegraphics[width=1\linewidth]{images/maingate_govt.jpg}
\end{figure}

\subsection{Infrastructure}
The school is located in 500 square metres area with 21 rooms in which only 6 rooms are usable including Headmaster's room. There is no staff room. The playground is in 0.2 acre of land which is quite less compared to the number of students in the School. There need to be much emphasis on building since it has been constructed for 40 years, and no renovation has been done for this building till now. \\ 

 \includegraphics[width=0.5\linewidth]{images/damage_room_1.jpg}
 \includegraphics[width=0.5\linewidth]{images/damage_room_2.jpg} \\

As we can see in the above figures the rooms are completely damaged. There is no hope for the school to depend on those classrooms. Government has not taken any steps to renovate or to build a new one. This keeps the school in the very last rank concerning the infrastructure and facilities. 

\subsection{Drinking water and Toilet facility}
More than 300 people staying there on the campus make drinking water and toilets the important facilities. Since it is a government school the infrastructure for sanitation was very poor. In 2019 an organisation called \textbf{YUVA Unstoppable} which is powered by \textbf{Step and Let's Solve} companies had come forward under their Corporate Social Responsibility (CSR) and they built 2 toilet rooms along with a drinking water tank. There are 2 toilets each for girls and boys and a water tank of 500 litres capacity. It will be filled once a week.

\begin{figure}[H]
    \centering
    \includegraphics[width=0.5\linewidth]{images/govt_YUVA.jpg}
    \caption{Toilet Built by YUVA Unstoppable Organization }
\end{figure}

\subsection{Students}
\begin{table}[H]
    \centering
    \begin{tabular}{c|c|c|c|c|c|c|c|c|c|c|c|c|c} \bottomrule \toprule 
          & \multicolumn{1}{c}{SC} & &  \multicolumn{1}{c}{ST} & & \multicolumn{1}{c}{Muslim} & &  \multicolumn{1}{c}{OBC} & & \multicolumn{1}{c}{General} & & \multicolumn{1}{c}{Total} & & Grand  \\ \bottomrule \toprule
         Class & B & G & B & G & B & G & B & G & B & G & B & G & Total \\ \bottomrule \toprule 
         1st & 3 & 2 & 3 & 7 & - & - & - & 3 & 4 & 2 & 10 & 14 & 24  \\ \hline 
         2nd & 5 & 3 & 2 & 6 & 3 & 2 & 1 & 1 & 2 & 3 & 13 & 15 & 28 \\ \hline 
         3rd & 3 & 1 & 3 & 6 & 3 & 5 & 2 & 1 & 1 & 1 & 12 & 14 & 26 \\ \hline
         4th & 3 & 4 & 8 & 12 & 1 & 2 & - & 1 & - & 3 & 12 & 22 & 34 \\ \hline
         5th & 8 & 6 & 3 & 7 & 3 & 2 & 4 & 2 & 1 & - & 19 & 17 & 36 \\  \hline 
         Total & 22 & 16 & 19 & 38 & 10 & 11 & 7 & 8 & 8 & 9 & 66 & 82 & 148 \\ \bottomrule \toprule 
         6th & 7 & 1 & 6 & 13 & - & - & 6 & 4 & 9 & 1 & 28 & 19 & 47 \\ \hline
         7th & 4 & 7 & 9 & 8 & 1 & 2 & 2 & 1 & 7 & 3 & 23 & 21 & 44 \\ \hline
         8th & 9 & 5 & 10 & 10 & 7 & 4 & 5 & 6 & 4 & 2 & 35 & 27 & 62 \\ \hline 
         Total & 20 & 13 & 25 & 31 & 8 & 6 & 13 & 11 & 20 & 6 & 86 & 67 & 153 \\ \hline 
         Grand Total & 42 & 29 & 44 & 69 & 18 & 17 & 20 & 19 & 28 & 15 & 152 & 149 & 301 \\ \bottomrule \toprule 
    \end{tabular}
    \caption{Number of students in every class divided based on the category}
\end{table}

There are 152 boys and 149 girls in a total of 301 students. We can see that number of girl students is almost similar to that of boys. But when we look closely there are no new admissions of girls, by which we can tell that parents decide about the girl students at a very small age. Either they are sent to study wholeheartedly are not at all exposed to academic culture.\\ [1\baselineskip]
This means that we should educate parents to send their girl children to school when the mother is lactating. We should tell them about the importance of Girl Child Education. Concerning Boys, there is a drastic increase in Admissions from class to class. 

\begin{figure}[H]
    \centering
    \includegraphics{images/graph.PNG}
    \caption{Class-wise strength of boys vs girls in the school}
\end{figure}

\subsection{Teachers}
There are 10 Teaching staff out of which 1 is male and the rest are female teachers. 4 non-teaching staff are appointed as Bisi Oota (Mid-day Meal) managing members. There is a need of attenders and cleaners in the school. Here is a list of teachers along with their qualification-

\begin{table}[H]
    \centering
    \begin{tabular}{c|c|c|c|c} \bottomrule \toprule 
        Sl. no & Teachers name & Qualification & Classes Handled & Subjects taken \\ \bottomrule \toprule 
        1 & Vrushbhendraiah & SSLC, D P.Ed & 1 to 10 & Physical Education \\ \hline 
        2 & Mrs. K Kalpana & B.Sc, B.Ed & 7,8 & Maths, Science \\ \hline 
        3 & Mrs. Shobha Devi & SSLC, TCH & 4,5 & Kannada, Maths, Science \\ \hline
        4 & Mrs. Manjula & PUC, TCH & 6 & Kannda, Science \\ \hline
        5 & Mrs. T Kanumakka & PUC, TCH & 5 & Maths, EVS, Kannada  \\ \hline
        6 & Mrs. P Anupama & BA, TCH & 7,8 & Social, Kannada \\ \hline
        7 & Mrs. Neelavati & BA, B.Ed & 6,7,8 & English \\ \hline
        8 & Mrs. Rajeshwari & BA, B.Ed & 1,2,3 & Nali - Kali \\ \hline 
        9 & Mrs. Kaccharavva & BA, D.Ed & 1,2,3 & Nali - Kali  \\ \hline
        10 & Mrs. Suvarna & BA, D.Ed & 1,2,3 & Nali - Kali  \\ \bottomrule \toprule 
    \end{tabular}
\end{table}

\subsection{Mid-Day Meals (MDM)}
Mid-Day Meal program is the Government of India's flagship program to achieve the Universalisation of Elementary Education (UEE) and is being implemented in partnership with State Government to cover the entire Karnataka. \\
Food is provided every afternoon for both students and teachers. The images given below show the quantity of the ingredients to be added to the food and the schedule of the food items provided. \\

    \includegraphics[scale=0.06]{images/MDM.jpg} \hfill
    \includegraphics[scale=0.075]{images/MDM_2.jpg}
    
\subsection{Teaching and Scheme of Evaluation}
\begin{enumerate}
    \item From Class 1 to 3 \textbf{Nali - Kali} system of Teaching is being followed. The phrase Nali - Kali means \textbf{learn while we play}. In these classes, teachers are just facilitators. Students are allowed to gather basic learning skills on their own. In this system, students are once explained the topic and then divided into groups. \\
    They talk to each other, sort the things out and learn the skills on their own. The syllabus basically consists of numbers and tables in Mathematics, alphabets and words in English, and Kannada, coloring and shape recognition skills. Just because to provide a suitable environment for them to discuss, they are divided into groups whose mother tongue is not similar and sitting desks are also not provided.
    
    The evaluation process is also much simpler than their learning. \textbf{Daily tasks} are given to students. For example this day they have to learn how to write their name. Although the teacher is present there, she will not involve much. Once the student confirms that s/he can write his name, the teacher asks her/him to write it on the board (this will reduce their public fear at the root level). \\
    Once s/he completes that, teacher marks that the student has completed the task. Based on the number of tasks completed students are given the final marks at the end of the semester. 
    
    \item From Class 4 to 8 \textbf{CCE} (Continuous and Comprehensive Evaluation) is followed. This type of teaching system is the traditional one and there will be some important aspects like lesson plan, teaching hours, scheme of evaluation and etc. Every subject teacher will submit his/her lesson plan before each and every class and get it corrected by the headmaster. \\
    Lessons are taught interactively. There is no ICT integration in this school as of now. Traditional Blackboard teaching is used in this school. The teacher arranges a unit test after completion of each and every lesson. This is considered as one of the best effective methods in CCE learning to evaluate the student. In final report card it is given some importance.
    
    \item Apart from this there will be 2 FAs (Formative Assessment) and 1 SA (Summative Assessment) in a semester which are much important than class tests. 
    \begin{enumerate}
        \item For Class 1 to 5
        \begin{itemize}
            \item FA 1 of 15 marks
            \item FA 2 of 15 marks 
            \item SA 1 of 20 marks
        \end{itemize}
        In report same weightage is given as corresponding to the marks attained by student. 
        \item For Class 6 to 8
        \begin{itemize}
            \item FA 1 of 10 marks
            \item FA 2 of 10 marks 
            \item SA 1 of 30 marks
        \end{itemize}
          The same system is followed in semester 2 where FA 1 and FA 2 are called FA 3 and FA 4 and SA 1 will be called SA 2.
    \end{enumerate}
    \end{enumerate}
    
\subsection{Exam and Evaluation}
\begin{enumerate}
\item Exam is \textbf{Conducted Based on Continuous and Comprehensive 
Evaluation} (CCE) system. Blueprint is made by subject teachers and 
later approved by the Headmaster for Summative Assessments at the end of 
the semester (SA 1 and SA 2).
\item Invigilators will be teachers of subjects other than the subject of the test. Seating arrangements are made in such a way that in a bench 3 
students are allowed to sit who are from different classes. 
\item Subject teachers set the question paper and get it approved by Headmaster before the exam. 
\end{enumerate}

\subsection{Report Card} 
All government schools in Karnataka have the same Report card layout. 1 st Page of the Report card includes details of the students such as name, class, date of birth, admission number, parents name, address, and students' photo. It also includes height, weight, and blood group of students. \\
School Details such as the Academic year and School Registration code will also be included. In next Page, it contains the academic achievements of the student which has subject wise break-up and their respective grades. 

\begin{figure}[H]
    \centering
    \includegraphics[width=0.9\linewidth]{images/report_card_govt.jpg}
    \caption{Report Card}
\end{figure}

Achievements in Co-Curricular activities are also included as per the Grading system mentioned in this Report. At the end of the Report, there will be mentioning of student's promotion to next class. It will be duly signed and sealed by School Headmaster and Will be distributed to students.

\subsection{Revision and Remedial Classes}
Wherever a student is lagging behind in academics, the teacher recognizes them and makes a list. Later these students are counselled by the respective subject and class teachers. If they find that there are Academic Related Issues, they will go for revision classes. All these procedures will happen during the Bridge Courses. If a student is found to be not that responsive then there will be Special Remedial Classes after school hours to improve his/her Academic Performance. 

\subsection{School Development Monitoring Committee (SDMC)}
\begin{enumerate}
\item To ensure community ownership and community participation in education, the Government have evolved this system of having a School Development and Monitoring 
Committee for each and every Government school. 
\item Apart from others, the main members of this committee will be 9 parents whose children are studying in the given school. 
\item The SDMC has been given necessary powers and functions for ensuring that the schools are managed better. Most of the issues relating to the academic aspects and developmental activities of the schools are addressed by SDMCs. 
\item The Head Master of the school functions as the Secretary of the SDMC.
\item The present system has been evolved by the issue of executive orders of the Government of Karnataka, and action is underway to continue the system of SDMC and even make it more effective. A proposal to make these SDMCs as part of the Panchayati Raj Institutions - by making them sub-committees of the respective Gram Panchayaths - is under consideration by the Government, in coordination with the Rural Development and Panchayath Raj Department.
\end{enumerate}
\begin{figure}
    \centering
    \includegraphics[width=1\linewidth]{images/sdmc_govt.jpg}
    \caption{Teachers getting trained by SDMC Officials}
\end{figure}

\subsection{Role of Community Participation in School Functioning}
\textbf{Parents and Teacher’s Interaction} -  After each Assessment is completed, there will be a Parent-teachers interaction to inform parents about their wards' academic progress. Most parents won’t attend this. So teachers will visit those students' homes whose academic performance is very low and inform parents the same.
\subsection{Events and Sports}
Sports are given a lot of importance in this school. As there is one more government school nearby this school children use their playground for all the sports activities. There have not been any such remarkable sports achievements but there is a lot of emphasis on Physical Education in this school. 

Science events are also an important part of this school. Some of the events are 
\begin{enumerate}
    \item \textbf{Pratibha Karanji} \\
    This event is organized once a year by the Government of Karnataka to encourage Co-Curriculum Activities for the students. There will be many competitions like Clay Modelling, Fancy Dress, Dance, Singing and many more. Students from this school have achieved district-level prizes.
    \item \textbf{TLM Mela} - Teaching Leaning Materials Mela \\
    This event is conducted by teachers to give insight into the teaching materials they use and how they are used. Especially science and math-related models are showcased here. Students also prepare some of the Models and there will be competition for that. Teachers train them and INSPIRE Award is given to winners from the government.  
\end{enumerate}

\subsection{Government schemes}
\begin{enumerate}
    \item Vitamin tablets like Pholic acid, albendazole are provided to students to fight malnutrition among them.
    \item Once a month, there will be a health check-up by a government ENT specialist. The check-up includes ENT, Tooth cavity, height, and weight. 
    \item  Minority students get Scholarships from the Government. 
\end{enumerate}
\subsection{Dropout Students}
Dropping out means leaving high school, college, university, or another group for practical reasons, necessities, or disillusionment with the system from which the individual in question leaves. \\
The School Headmaster and BEO of that sector will take the initiative in bringing back those children to school. Those students are recovered from home after a long absence. They are trying to attend class right now. Department is striving very hard to get those students to normal streamline. 
\subsection{Schools in the times of pandemic}
\begin{itemize}
    \item This novel pandemic disrupted the normal functioning of schools and their activities. Due to lockdowns schools were shut. So new methodologies were adopted to facilitate the teaching and assessment of students. Online classes and assessments replaced conventional learning.
    \item Private schools did all the academic activities in virtual mode - conducting classes, submission of assignments in typed/scanned format, etc. Those students who couldn't afford smartphones faced problems by this.
    \item Meanwhile, the government started \textbf{Vidyagama} and \textbf{Vathara Shaale} initiatives along with online classes which helped poor children who can't afford electronic gadgets. 
    \item Teachers went to students' houses, students and teachers gathered in open public places like temples, groupings near home, and academic activities were done. 
    \item For lower classes teachers used to assign some works to students there and marks were awarded for the same whereas, for higher classes, activities and assignments were given based on the syllabus for formative and summative assessments which students submitted in the next class. Marks were awarded for the same.
    \item  These internal marks along with student's previous year's academic performance was used to give results. 
    \item About online classes - Some schools conducted online classes on various platforms like zoom, Google meets, Microsoft teams, DD Chandana classes were also there.
\end{itemize}

In some schools, formative assessments were based on activities and summative assessments were based on assignments. On the other hand, some schools used internal marks for activities and precious year's performance to give the final result of students. SSLC exams were conducted and All students are passed. 