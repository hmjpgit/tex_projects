\section{St. Paul`s English High School}

\subsection{Mission}
The mission at St. Paul`s Higher Primary School is,
\begin{itemize}
    \item To empower students to acquire, demonstrate, articulate and value knowledge and skills that will support them as life-long learners, 
    \item To participate in and contribute to the global world and practise the core values of the school: respect, tolerance and inclusion, and excellence.
\end{itemize}
\subsection{Infrastructure}
Total area of the school is 550 sq mt. The building is in the 2nd floor a commercial complex. Top floor is used as playground. There are 20 rooms which includes staff room, laboratory, waiting room, sports equipment`s store, Science lab, Computer lab. Concerning playground the location of the school there need some changes for the overall development of the students.\\ \\ 
\begin{figure}[H]
    \centering
    \includegraphics[width=1\linewidth]{images/image.png}
\end{figure}
\newpage 
\subsection{Students}
From class 1 to 10 there are 328 students in total, admissions are very less in this school due to the more suffocation in the building and many other reasons. But its very clean and maintained properly by non - teaching staff. The below list is the number of students class wise.
\begin{table}[H]
    \centering
    \begin{tabular}{c|c|c|c} \toprule \bottomrule 
         Class & Boys & Girls & Total   \\ \toprule \bottomrule  
         1st & 16 & 19 & 35 \\ \hline 
         2nd & 13 & 19 & 32 \\ \hline 
         3rd & 30 & 19 & 49 \\ \hline 
         4th & 15 & 20 & 35 \\ \hline 
         5th & 22 & 15 & 37 \\ \hline 
         6th & 27 & 17 & 44 \\ \hline 
         7th & 16 & 11 & 27 \\ \hline 
         8th & 20 & 12 & 32 \\ \hline 
         9th & 7 & 10 & 17 \\ \hline 
         10th & 11 & 9 & 20 \\ \hline 
         Total & 177 & 151 & 328 \\ \toprule \bottomrule 
    \end{tabular}
\end{table}

\subsection{Teachers}
There is one speciality in this school, anyone as a teacher who wants to teach in this school should compulsorily have Diploma in Education or Bachelors in Education. The reason behind this is that teacher can handle any sort of students if they know the student psychology, and it will be taught only in the above mentioned courses. Another reason behind this is to give quality output. They need qualified teachers and those teachers are the one`s who are specially trained to be teachers.\\

This is a list of teachers who teach from class 1 to 5, we can observe that most of them have Training as teachers.
\begin{table}[H]
    \centering
    \begin{tabular}{c|c|c|c}\bottomrule \toprule 
        Sl. no & Teachers name & Qualification & Subjects taken\\ \bottomrule \toprule
        1 & Mamta & MA D.Ed & Kannada \\ \hline 
        2 & Vani & BA B.Ed & Social \\ \hline 
        3 & Rehana & BA B.Ed & Hindi \\ \hline 
        4 & Shaina Banu & BSc D.Ed & Science \\ \hline 
        5 & Asma & B Com & Maths \\ \hline 
        6 & Iliyar & PUC D.CSE & Computer \\ \hline 
        7 & Chandini & BA & English \\ \bottomrule\toprule
    \end{tabular}
\end{table}
\newpage
This is list of teachers who teach from Class 6 to 10, including Umadevi as Coordinator for Online Classes and digital learning. 
\begin{table}[H]
    \centering
    \begin{tabular}{c|c|c|c}\bottomrule \toprule 
        Sl. no & Teachers name & Qualification & Subjects taken\\ \bottomrule \toprule
        1 & Meghana & BA B.Ed & English, Social \\ \hline 
        2 & Mehaboob Hussain & MA B.Ed & Hindi \\ \hline 
        3 & Hussain Holi & MA B.Ed & English, Social \\ \hline 
        4 & Vijaya & BSc B.Ed & Science \\ \hline 
        5 & Chaitra & B.Com D.CSE & Computer \\ \hline 
        6 & Girija & M.Com B.Ed & Maths \\ \hline 
        7 & Lakshmi & MA B.Ed & Kannada \\ \hline 
        8 & Huliganna & B.P.Ed & Sports\\ \bottomrule\toprule
    \end{tabular}
\end{table} 
\subsection{Exam and Evaluation}
As similar to all other schools there will be 2 FAs and 1 SA in a semester, same followed in 2nd semester too. The weightage of the marks is as follows
\begin{enumerate}
    \item FA 1 in July for 25 marks
    \item FA 2 in September for 25 marks
    \item SA 1 in November for 50 marks which is further divided as 40 marks for written test and 10 marks for oral Viva 
\end{enumerate}
\subsection{Reportcard}
\begin{figure}[H]
    \centering
    \includegraphics[width=0.75\linewidth]{images/report_cards.jpg}
\end{figure} This is the example of the report card. We can see the marks distribution and scheming process. \\ 

Overall there need to be lot of improvement in this school in regards of Infrastructure, Physical Education, Co - Curricular activities, Events and Basic Facilities. 
